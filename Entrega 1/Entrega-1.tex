\documentclass{article}

\usepackage{arxiv}

\usepackage[utf8]{inputenc} % allow utf-8 input
\usepackage[T1]{fontenc}    % use 8-bit T1 fonts
\usepackage{lmodern}        % https://github.com/rstudio/rticles/issues/343
\usepackage{hyperref}       % hyperlinks
\usepackage{url}            % simple URL typesetting
\usepackage{booktabs}       % professional-quality tables
\usepackage{amsfonts}       % blackboard math symbols
\usepackage{nicefrac}       % compact symbols for 1/2, etc.
\usepackage{microtype}      % microtypography
\usepackage{graphicx}

\title{Control Estadístico de Calidad (Entrega 1)}

\author{
    Michel Mendivenson Barragán Zabala
   \\
    Departamento de Estadística \\
    Universidad Nacional de Colombia \\
   \\
  \texttt{\href{mailto:mbarraganz@unal.edu.co}{\nolinkurl{mbarraganz@unal.edu.co}}} \\
   \And
    Juan Sebastián Huertas Pirajan
   \\
    Departamento de Estadística \\
    Universidad Nacional de Colombia \\
   \\
  \texttt{\href{mailto:juhuertasp@unal.edu.co}{\nolinkurl{juhuertasp@unal.edu.co}}} \\
   \And
    Diego Andres Paez Molina
   \\
    Departamento de Estadística \\
    Universidad Nacional de Colombia \\
   \\
  \texttt{\href{mailto:dpaezm@unal.edu.co}{\nolinkurl{dpaezm@unal.edu.co}}} \\
  }


% tightlist command for lists without linebreak
\providecommand{\tightlist}{%
  \setlength{\itemsep}{0pt}\setlength{\parskip}{0pt}}



\setcounter{secnumdepth}{0}
\renewenvironment{abstract}{}{}
\renewcommand{\theenumi}{\alph{enumi}}
\begin{document}
\maketitle


\begin{abstract}

\end{abstract}


\hypertarget{ejercicio-1}{%
\subsection{Ejercicio 1}\label{ejercicio-1}}

Sea \(X \sim N(\mu, \sigma)\) una característica de calidad. Mediante
simulaciones Mediante simulaciones, establezca el comportamiento del
\emph{ARL} (en control y fuera de él) de las Cartas \(R\) y \(S\) para
observaciones normales con límites \(3\sigma\) y muestras de tamaño (a)
\(n = 3\) y (b) \(n = 10\) ¿Qué regularidades observa?

\begin{quote}
Para la implementación de la solución, se creará en R una función que
nos permita simular cuantás veces querramos el momento en un proceso da
una alerta (bien sea verdadera o falsa) con argumentos que nos permitan
modificar tanto el tamaño de muestra \(n\) como los límites de la carta
de control y su línea central. La función se define como sigue:
\end{quote}

\begin{quote}
Los resultados para diferentes corrimientos de la línea central se
encuentran condensados en la tabla a continuación:
\end{quote}

\begin{quote}
Como podemos observar \(\cdots\)
\end{quote}

\hypertarget{ejercicio-2}{%
\subsection{Ejercicio 2}\label{ejercicio-2}}

Sea \(X \sim N(\mu, \sigma)\) una característica de calidad. Se sabe que
los valores objetivo de los parámetros del proceso son \(\mu = \mu_0\) y
\(\sigma = \sigma_0\). Construir las curvas OC de la carta \(S^2\) con
límites de probabilidad. Interpretar los resultados

\hypertarget{ejercicio-3}{%
\subsection{Ejercicio 3}\label{ejercicio-3}}

Sea \(X \sim N(\mu_0, \sigma_0)\) una característica de calidad.
Construya la carta \(\bar{X}\) para el monitoreo de la media del
proceso. Genere 10 muestras de tamaño \(n\) provenientes de \(X\), de
tal modo que la media muestral de ninguna de ellas caiga fuera de los
límites de control. A partir del undécimo momento de monitoreo se pide
generar muestras del mismo tamaño \(n\) provenientes de una distribución
normal con media \(\mu_1 = \mu_0 + k\sigma_0\) y \(\sigma_1 = \sigma_0\)
(con \(k = 1,0\)) hasta que la carta emita una señal por primera vez. Si
se asume que el proceso caracterizado por \(X\) es estable y que se
desconoce el momento en el cual se produjo el incremento en el nivel
medio, ¿en qué muestra ocurrió el cambio en la media del proceso más
probablemente?

\hypertarget{ejercicio-4}{%
\subsection{Ejercicio 4}\label{ejercicio-4}}

Sea \(X \sim N(\mu_0, \sigma_0)\) una característica de calidad. Se
pide:\\

\begin{enumerate}
  \item Mediante simulaciones, establezca el comportamiento del ARL de la Carta $\bar{X}$ con límites tres sigma para observaciones normales.
  \item Genere 20 subgrupos racionales de tamaño $n = 3$ provenientes de $X$. Asúmase que el proceso es estable en cuanto a dispersión y con los subgrupos iniciales, construya la carta $\bar{X}$ como es habitual hasta verificar la estabilidad del proceso. Establezca el comportamiento del ARL para la carta que se obtiene del análisis de Fase I realizado.
  \item Repetir lo indicado en el literal (b) con 50 subgrupos racionales de tamaño $n = 3$. Comente los resultados.
\end{enumerate}

\hypertarget{ejercicio-5}{%
\subsection{Ejercicio 5}\label{ejercicio-5}}

Calcular el ARL de la Carta \(\bar{X}\) mediante cadenas de Markov.
Diseñar la carta con límites de control ubicados a tres desviaciones
estándar de la media y dividiendo la región de control estadístico en
franjas de ancho igual a una desviación estándar.

\bibliographystyle{unsrt}
\bibliography{}


\end{document}
